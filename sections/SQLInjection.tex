\section{SQL Injection}
Als SQL Injection wird das Einschleusen von eigenen SQL Befehlen in die Datenbank einer Web Applikation bezeichnet. Ziel des Angriffs ist meist Zugriff auf die Datenbank oder die angegriffene Web Applikation zu erhalten um Daten zu stehlen, nach eigenem Interesse zu ver\"andern, gr\"o{ss}tm\"oglichen Schaden anrichten oder einfach die Verf\"ugbarkeit der Anwendung zu behindern. SQL Injections k\"onnen durchgef\"uhrt werden, wenn eine Website die Usereingaben unzureichend filtert.\cite{hackingWebAppsBuch}

\subsection{Grundlagen}
Um einen erfolgreichen Angriff dieser Art durchf\"uhren zu k\"onnen, muss erstmal eine verwundbare Website gefunden werden. Der einfachste Weg daf\"ur ist bekannt als ``Google Dorking''. Dabei werden spezielle Suchbegriffe in Google verwendet um potentiell verwundbare URLs aufzusp\"uren. Ein solcher Suchbegriff k\"onnte beispielsweise folgenderma{\ss}en aussehen: \texttt{inurl:index.php?id=} \cite{sqlinjectionTutorial}\\
Der Umfang einer SQL Injection ist also nur limitiert durch die Kreativit\"at des Angreifers, die Zeichen die von der Seite akzeptiert werden und welche Zeichen ausgefiltert werden.
\cite{sqlinjectionTutorial}

\subsection{Funktionsweise}
Wenn eine Web Applikation SQL Statements aus Anfragedaten bildet, verwendet diese \"ublicherweise die Eingaben des Users als Zahlen oder Strings. Wenn man nun an dieser Stelle die Query durch z.B. ein einfaches Hochkomma fr\"uhzeitig abbricht, bekommt man im besten Fall von der Applikation sogar eine Fehlermeldung zur\"uck. Dies deutet zumindest auf schlechtes Inputfiltern und unzureichendes Error Handling hin. Ist das nicht der Fall, wird das SQL Injecting wesentlich erschwert aber allein dadurch noch keinesfalls unm\"oglich gemacht.

\clearpage
\subsection{Effekte}
Es k\"onnen verschiedene Effekte mithilfe von SQL Injection erzielt werden.
\begin{itemize}
\item Umgehen von Authentifizierung\\
Diese Attacke erlaubt es dem Angreifer sich (m\"oglicherweise sogar mit Admin Privilegien) einzuloggen, ohne g\"ultige Login Daten anzugeben.\\
\item Daten offenlegen\\
Hier ist das Ziel, sensible Daten aus einer Datenbank zu stehlen.\\
\item Kompromittieren der Integrit\"at von Daten\\
Das Ziel ist, Daten gezielt zu ver\"andern um beispielsweise Malware in eine Website einzuschleusen.\\
\item L\"oschen von Daten\\
Hier geht es meist einfach nur darum Schaden anzurichten. Dies kann aus verschiedenen Gr\"unden geschehen.\\
\item ``Remote Command Execution''\\
Dies kann dem Angreifer Zugriff auf das darunterliegende Betriebssystem gew\"ahren.
\end{itemize}

\cite{ciscoSQLInjection}

\subsection{Schutzma{\ss}nahmen}
Die Ma{\ss}nahmen gegen SQL Injection, sind sehr \"ahnlich derer gegen XSS. Im folgenden sind diese kurz zusammengefasst.
\begin{itemize}
\item Daten zu einem einheitlichen Zeichensatz (z.B. UTF-8) normalisieren
\item Daten mit erwarteten Datentypen abgleichen
\item Daten mit erwartetem Inhalt abgleichen (z.B. g\"ultige Postleitzahl, ...)
\item Ung\"ultige Eingaben komplett ablehnen und nicht versuchen fehlerhafte Werte selbst zu bereinigen
\end{itemize}
\cite{hackingWebAppsBuch}

\clearpage
\subsection{Beispie}
Wenn man zum Beispiel ein Login Formular hat:

\begin{lstlisting}[caption = Login Formular]
<form action="/cgi-bin/login" method=post>
 Username: <input type=text name=username> 
 Password: <input type=password name=password> 
<input type=submit value=Login>
\end{lstlisting}

Wird normalerweise der Username und das Passwort in einem HTTP POST Request folgenderma{\ss}en \"ubermittelt.

\begin{lstlisting}[caption = Normaler HTTP POST Request]
username=submittedUser&password=submittedPassword
\end{lstlisting}

Dies h\"atte folgende SQL Abfrage zur Folge.

\begin{lstlisting}[caption = Normale SQL Query]
select * from Users where (username = 'submittedUser' and password = 'submittedPassword');
\end{lstlisting}

Wenn nun allerdings der Userinput nicht ordentlich bereinigt wird kann man sich mit folgender Anfrage ohne Passwort einloggen.

\begin{lstlisting}[caption = Modifizierter HTTP POST Request]
username=admin%27%29+--+&password=+
\end{lstlisting}

Dadurch wird eine im eigenen Interesse ver\"anderte SQL Abfrage gestartet.

\begin{lstlisting}[caption = Modifizierte SQL Query]
select * from Users where (username = 'admin') -- and password = ' ');
\end{lstlisting}

Dabei wird die Abfrage des Passworts mithilfe der beiden Bindestriche (``--``) auskommentiert und dadurch nicht mehr \"uberpr\"uft.

\cite{ciscoSQLInjection}